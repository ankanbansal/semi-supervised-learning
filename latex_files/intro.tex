\section{Introduction}

% This is all just placeholder text.
Data-hungry deep neural networks require vast amounts of labeled data. Collecting images from the
internet has never been easier. However, getting these images annotated is expensive and
time-consuming. We need to devise algorithms that can use the unlabeled data as well. This unlabeled
data can be utilized in different ways. In computer vision, these methods can broadly be divided
into i.) unsupervised learning, ii.) weakly-supervised learning, and iii.) semi-supervised learning.

In unsupervised learning, the aim is to learn discriminative representations from data without any
supervision from annotations. In weakly-supervised learning, the annotations are provided at a
coarser-level than the task. For example, for object detection, they could be just image-level
annotations. Semi-supervised learning is the most general of these settings. Here, annotations for
some data while a large portion of the data is un-annotated. There are several variants of
semi-supervised learning. These include webly-supervised learning \cite{} and omni-supervised
learning \cite{}. 

In this paper, we present an approach for semi-supervised image classification. The proposed
approach uses two clustering penalties. The first one (Mean Entropy Loss (MEL)) encourages the
classifier to be more confident about its prediction and the second (Negative Batch Entropy Loss
(NBEL)) incorporates prior knowledge about the class distribution of the dataset into the learning
pipeline. These two penalties can be applied along with the cross-entropy loss commonly used for
fully-supervised learning. The benefit of using these penalties is that they can be applied to both
supervised and unsupervised data. 

The presented approach can be... 




%\begin{figure}[t]
%\begin{center}
%\fbox{\rule{0pt}{2in} \rule{0.9\linewidth}{0pt}}
%   %\includegraphics[width=0.8\linewidth]{egfigure.eps}
%\end{center}
%   \caption{Example of caption.  It is set in Roman so that mathematics
%   (always set in Roman: $B \sin A = A \sin B$) may be included without an
%   ugly clash.}
%\label{fig:long}
%\label{fig:onecol}
%\end{figure}
%
%
%\begin{figure*}
%\begin{center}
%\fbox{\rule{0pt}{2in} \rule{.9\linewidth}{0pt}}
%\end{center}
%   \caption{Example of a short caption, which should be centered.}
%\label{fig:short}
%\end{figure*}
